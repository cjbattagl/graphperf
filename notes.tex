\documentclass[11pt]{article}

\usepackage{amssymb}
\usepackage{amsmath}
\usepackage{graphicx}
\usepackage{cite}
\usepackage{algorithmic}
\usepackage{algorithm}
\usepackage{todonotes}
\usepackage{url}

\usepackage{listings}
 \lstset{
            language=Matlab,                                % choose the language of the code
    %       basicstyle=10pt,                                % the size of the fonts that are used for the code
            numbers=left,                                   % where to put the line-numbers
            numberstyle=\footnotesize,                      % the size of the fonts that are used for the line-numbers
            stepnumber=1,                                           % the step between two line-numbers. If it's 1 each line will be numbered
            numbersep=5pt,                                  % how far the line-numbers are from the code
    %       backgroundcolor=\color{white},          % choose the background color. You must add \usepackage{color}
            showspaces=false,                               % show spaces adding particular underscores
            showstringspaces=false,                         % underline spaces within strings
            showtabs=false,                                         % show tabs within strings adding particular underscores
    %       frame=single,                                           % adds a frame around the code
    %       tabsize=2,                                              % sets default tabsize to 2 spaces
    %       captionpos=b,                                           % sets the caption-position to bottom
            breaklines=true,                                        % sets automatic line breaking
            breakatwhitespace=false,                        % sets if automatic breaks should only happen at whitespace
            escapeinside={\%*}{*)}                          % if you want to add a comment within your code
}

\setlength{\paperwidth}{8.5in}
\setlength{\paperheight}{11in}
\setlength{\voffset}{-0.2in}
\setlength{\topmargin}{0in}
\setlength{\headheight}{0in}
\setlength{\headsep}{0in}
\setlength{\footskip}{30pt}
\setlength{\textheight}{9.25in}
\setlength{\hoffset}{0in}
\setlength{\oddsidemargin}{0in}
\setlength{\textwidth}{6.5in}
\setlength{\parindent}{0in}
\setlength{\parskip}{9pt}

\newcommand{\ben}{\begin{enumerate}}
\newcommand{\een}{\end{enumerate}}

\DeclareGraphicsRule{.JPG}{eps}{*}{`jpeg2ps #1}

\title{Power-Law Graphs and Partitioning: Notes}
\author{Casey Battaglino}
\date{}
\begin{document}
\maketitle
\tableofcontents

\section{How important is balance? (vs. edge cut)}
Inspiration: ``Myth 3: Partition Quality is Paramount''

i.e, consider an amortization case and a dynamic partitioning case

\subsection{Shared-memory Spmv Performance}
Can we use the generating function approach to determine characteristics of the matrix in RCM order? (i.e, bandwidth)... (BFS connection) ... and then apply this to a performance model using the `amount of band in cache' model...

Can apply to a variety of degree distributions:


\subsection{Distributed-memory Spmv Performance}
Base strategies: hashing (random).
Two partition strategies: streaming model or an offline model. In either case, can we produce a principled approach to estimating edge-cut size? Produce performance model based on edge-cut size on a distributed-memory grid. (Well.. vertices on boundary)

BFS communication volume perfectly matches hypergraph cut \cite{Catalyurek_hypergraph-partitioningbased}

\subsection{Deeper models}
Vertices on boundary vs. edge cut -- is there a disparity here?

Minimum-congestion partitioning?

\section{Random graph model flaws / Gaming the Graph500}
Compare degree-sorted real-world network with Kronecker, show differences.

\section{Where does streaming partitioning fail?}
\subsection{Probabilistic Argument}
\paragraph{$C_n$ (or $P_n$)} ($OPT=2=O(1)$)

\textit{Adversarial ordering}: $\{ \{ 1,3, \cdots , n-1\},\{ 2, 4, \cdots, n \}\}$. 

No edges revealed until $\frac{n}{2}$ edges traversed, so cannot perform better than $\frac{n}{2}$ cut in expectation.

\textit{Random ordering}: 

$\Pr[t $ arrives with no edges$]  =  \Pr[$both neighbors arrive after $ t ]  = \frac{n-t}{n}\frac{n-t-1}{n-1}$ 

$E[\# $ with no edge$] \approx \frac{n}{3}$

\section{Streaming partitioning and input order}
What characteristic order do we expect in the real world, with dynamically forming graphs (ie preferential attachment)?

BFS/DFS order modeling

Is there a way of doing hashed partitioning on a grid?

\section{FENNEL/Heuristics via Spmv kernels}
Can we give a $o(n)$ proof for power-law graphs? Or for some ordering? \cite{DBLP:journals/corr/abs-1212-1121}. ``An interesting open question is to develop techniques for analyzing streaming graph algorithms on BFS and DFS orders.''

That is, for vertex $t$, give closed form for $\Pr[t $ arrives with no edges$]$

Streaming algorithm followed by an spmv-style partition reordering? How about map-reduce?

\section{Multilevel partitioning via spmv kernels}
Existing work: hypergraph partitioning

Pegasus: \cite{kang2009pegasus}


\section{1D vs 2D BFS}
\cite{Buluc:2011:PBS:2063384.2063471}

\section{Vertex-Local Clustering for Power-Law graphs}
\cite{Andersen:2009:FSC:1536414.1536449}

\lstinputlisting[language=Matlab, caption=Independent Set Computation]{indset.m}

\lstinputlisting[language=Matlab, caption=Peer Pressure Clustering]{peerpress.m}
\begin{verbatim}

\end{verbatim}

\begin{verbatim}

\end{verbatim}

\section{Hypergraph Partitioning for Power-Law graphs}
Represent an abstract power-law model within the hypergraph model... what does it look like

\section{Appendix: Random Graph Models}
\subsection{Erdos-Renyi / Poisson}

\subsection{Kronecker}
\cite{leskovec2007scalable}

\subsection{Chung-Lu}
\cite{chung2004average}

\subsection{Hidden-Partition}
\cite{Condon99algorithmsfor}

\subsection{Watts-Strogatz}
\cite{Watts1998}

\section{Appendix: Graph Motifs}
\subsection{Triangle Counting}


\section{Appendix: Graph-Partitioning strategies}

\subsection{Integer Program}
\subsubsection{Cheeger's Inequality}
\subsubsection{Minbis QIP}
\subsubsection{SDP-3 (Goemans; Arora, Rao, Vazirani)}
\cite{arora2009expander}

\subsubsection{SDP-2, ``Vector Relaxation'' (Goemans, Williamson)}
\cite{goemans1995improved}

\subsubsection{SDP-1, ``Spectral Relaxation,'' (Fiedler, Alon et al)}
\cite{alon1986eigenvalues}

\subsubsection{LP, Multicommodity Flow (Leighton-Rao)}
\cite{Leighton:1999:MMM:331524.331526}

\subsection{Hypergraph Partitioning}
Hypergraph streaming?
\cite{catalyurek1999hypergraph}

\subsection{Streaming Graph Partitioning}
\subsubsection{Stanton-Kliot}
\cite{DBLP:journals/corr/abs-1212-1121}
\cite{Stanton:2012:SGP:2339530.2339722}

\subsubsection{Fennel}
\cite{tsourakakis2012fennel}

\subsection{Multilevel Graph Partitioning}
\subsubsection{METIS}
\cite{karypis1998multilevel}


%\begin{figure}
%\label{rpls}
%\center \includegraphics*[width=5in]{mat.png}
%\end{figure}

\bibliographystyle{plain}
\bibliography{bib}


\end{document}
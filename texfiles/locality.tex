\documentclass[11pt]{article}

\usepackage{amssymb}
\usepackage{amsmath}
\usepackage{graphicx}
\usepackage{cite}
\usepackage{algorithmic}
\usepackage{relsize}
\usepackage{algorithm}
\usepackage{todonotes}
\usepackage{url}
\usepackage{tikz}
\usetikzlibrary{arrows}

\usepackage{listings}
 \lstset{
            language=Matlab,                                % choose the language of the code
    %       basicstyle=10pt,                                % the size of the fonts that are used for the code
            numbers=left,                                   % where to put the line-numbers
            numberstyle=\footnotesize,                      % the size of the fonts that are used for the line-numbers
            stepnumber=1,                                           % the step between two line-numbers. If it's 1 each line will be numbered
            numbersep=5pt,                                  % how far the line-numbers are from the code
    %       backgroundcolor=\color{white},          % choose the background color. You must add \usepackage{color}
            showspaces=false,                               % show spaces adding particular underscores
            showstringspaces=false,                         % underline spaces within strings
            showtabs=false,                                         % show tabs within strings adding particular underscores
    %       frame=single,                                           % adds a frame around the code
    %       tabsize=2,                                              % sets default tabsize to 2 spaces
    %       captionpos=b,                                           % sets the caption-position to bottom
            breaklines=true,                                        % sets automatic line breaking
            breakatwhitespace=false,                        % sets if automatic breaks should only happen at whitespace
            escapeinside={\%*}{*)}                          % if you want to add a comment within your code
}

\setlength{\paperwidth}{8.5in}
\setlength{\paperheight}{11in}
\setlength{\voffset}{-0.2in}
\setlength{\topmargin}{0in}
\setlength{\headheight}{0in}
\setlength{\headsep}{0in}
\setlength{\footskip}{30pt}
\setlength{\textheight}{9.25in}
\setlength{\hoffset}{0in}
\setlength{\oddsidemargin}{0in}
\setlength{\textwidth}{6.5in}
\setlength{\parindent}{0in}
\setlength{\parskip}{9pt}

\newcommand{\ben}{\begin{enumerate}}
\newcommand{\een}{\end{enumerate}}

\DeclareGraphicsRule{.JPG}{eps}{*}{`jpeg2ps #1}

\title{Maximizing Locality of Graph Traversals}
\author{Casey Battaglino}
\date{}
\begin{document}
\maketitle
%\tableofcontents

\section{Breadth-First Traversal}

Given a graph $G$ with adjacency matrix $A \in \mathbb{R^{(N \times N)}}$, a frontier expansion of a breadth-first search is equivalent to the sparse matrix-vector product $A^Tx$, where $w$ contains a nonzero for each element in the current reach. 

If $A$ is represented in CSR format, the number of flops required is $2 \times nnz(A)$. However, only those rows in $A$ that correspond to nonzero indices in $x$ need be loaded from memory. 

\subsection{Replacement Policies}
\cite{Dan:1990:AAL:98457.98525}

\subsection{Tuning}
\cite{Xia_topologicallyadaptive}

\subsection{Cache Model}
When a row is loaded from memory, its nonzeros are stored in cache according to some cache policy. Assume for now that we are computing on a single core with multiple levels of cache. After a particular traversal step $i$, we can imagine a permutation $p_i \in \Sigma_n$ that orders the rows of $A$ in the order in which they are stored in cache. 

Now consider the traversal step $i+1$: based on the particular cache policy, we can imagine a recurrence relation that describes $p_{i+1}=f(p_i; A, x_i)$

\subsection{Notes}
\begin{enumerate}
\item What does the spectrum tell us about the growth of $x_i$?
\item What does the degree distribution tell us (gen function methodology)?
\item Kronecker - PL mapping
\end{enumerate}




\bibliographystyle{plain}
\bibliography{bib}


\end{document}

\documentclass[11pt]{article}

\usepackage{amssymb}
\usepackage{amsmath}
\usepackage{graphicx}
\usepackage{cite}
\usepackage{algorithmic}
\usepackage{relsize}
\usepackage{algorithm}
\usepackage{todonotes}
\usepackage{url}
\usepackage{tikz}
\usetikzlibrary{arrows}

\usepackage{listings}
 \lstset{
            language=Matlab,                                % choose the language of the code
    %       basicstyle=10pt,                                % the size of the fonts that are used for the code
            numbers=left,                                   % where to put the line-numbers
            numberstyle=\footnotesize,                      % the size of the fonts that are used for the line-numbers
            stepnumber=1,                                           % the step between two line-numbers. If it's 1 each line will be numbered
            numbersep=5pt,                                  % how far the line-numbers are from the code
    %       backgroundcolor=\color{white},          % choose the background color. You must add \usepackage{color}
            showspaces=false,                               % show spaces adding particular underscores
            showstringspaces=false,                         % underline spaces within strings
            showtabs=false,                                         % show tabs within strings adding particular underscores
    %       frame=single,                                           % adds a frame around the code
    %       tabsize=2,                                              % sets default tabsize to 2 spaces
    %       captionpos=b,                                           % sets the caption-position to bottom
            breaklines=true,                                        % sets automatic line breaking
            breakatwhitespace=false,                        % sets if automatic breaks should only happen at whitespace
            escapeinside={\%*}{*)}                          % if you want to add a comment within your code
}

\setlength{\paperwidth}{8.5in}
\setlength{\paperheight}{11in}
\setlength{\voffset}{-0.2in}
\setlength{\topmargin}{0in}
\setlength{\headheight}{0in}
\setlength{\headsep}{0in}
\setlength{\footskip}{30pt}
\setlength{\textheight}{9.25in}
\setlength{\hoffset}{0in}
\setlength{\oddsidemargin}{0in}
\setlength{\textwidth}{6.5in}
\setlength{\parindent}{0in}
\setlength{\parskip}{9pt}

\newcommand{\ben}{\begin{enumerate}}
\newcommand{\een}{\end{enumerate}}

\DeclareGraphicsRule{.JPG}{eps}{*}{`jpeg2ps #1}

\title{Matrix SSSP on an Evolving Graph}
%\author{Casey Battaglino}
\date{}
\begin{document}
\maketitle
%\tableofcontents

\section{Breadth-First Traversal for SSSP}

Given a graph $G$ with adjacency matrix $A \in \mathbb{R^{(N \times N)}}$, consider the problem of computing a shortest path from start vertex $s$ using a breadth-first traversal. 

\subsection{Min-Plus Algebra}
Consider the $\min,+$ semi-ring, also known as the tropical ring: $(\mathbb{R} \cup \{\infty\},\min,+,\infty,0)$. The operations are thus assigned $\oplus \gets \min$ and $\otimes \gets +$. Matrix-vector multiplication is thus defined as \begin{align}Ax = \bigoplus_j a_{ij} \otimes x_j = \min_j\{a_{ij}+x_j\}\end{align}

In a step of the Dijkstra SSSP algorithm, we wish to have a vector which contains the computed distances from the source vertex to visited vertices. The initial state is:

\begin{align}x_0^T = \bordermatrix{ 
& & & & s&  \cr
& \infty, & \infty, & \cdots & 0,& \cdots &\infty \cr
} \end{align}

The incidence matrix will contain a weight value for each edge in $G$, 0 for each diagonal element, and $\infty$ for each absent edge.


\begin{align*}
\begin{bmatrix}
0 & w_{12} & w_{13}  \cr
w_{21} & 0 &  \infty   \cr
\infty& w_{32} & 0   
\end{bmatrix}\end{align*}

\begin{align*}
\begin{tikzpicture}[->,>=stealth',shorten >=1pt,auto,node distance=3cm,
thick,main node/.style={circle,draw,font=\sffamily\Large\bfseries}]
  \node[main node] (1) {1};
  \node[main node] (2) [below left of=1] {2};
  \node[main node] (3) [below right of=1] {3};
  \path[every node/.style={font=\sffamily\small}]
    (1) edge node [left] {$w_{13}$} (3)
        edge [bend left] node[left] {$w_{12}$} (2)
    (2) edge [bend left] node {$w_{21}$} (1)
    (3) edge node {$w_{32}$} (2);
\end{tikzpicture}\end{align*}

Following Algebraic Dijkstra, a single frontier-expansion of the traversal involves computing $x_{i+1} = Ax_i$ \cite{Gazit198861,citeulike:10648615}. This recurrence can be expressed in closed form as $x_{i+1} = (A)^i x_0$.

A standard storage format for sparse matrices is Compressed Sparse Row (CSR) or Compressed Sparse Column (CSC). The $\infty$ and 0 values of the incidence matrix can be safely ignored by this representation and the algebraic product will still compute the distance vector, thus we need only $O(|E|)$ storage.

Insertion of edges into CSC or CSR format is difficult because nonzeros are stored in a compact column- or row-index array. Thus we demonstrate how to handle an evolving graph algebraically with an update matrix. 

\subsection{Vertex, Edge Insertion}
Consider the case where we have some set of vertices $V'$ and edges $E'$ to add to the graph $(G' \gets G \cap \{V',E'\})$. These are formed as a matrix update, such that the updated graph $A'$ is represented by $A' \gets A + A_u$. Note that matrix addition in this semi-ring is the element-wise $\min$ operation.

In the case that the matrix updates often, we may not want to compute $A'$. In this case, we can maintain $A_u$ as a more dynamic data structure, such as unordered or ordered triples, either of which has $O(nnz(A_u))$ cache complexity for SpMV. \cite{sparsechapter}.

\begin{enumerate}
\item To deal with vertex insertion, we must simply ensure that $A,A_u \in \mathbb{R}^{(N+|V'|)\times(N+|V'|)}$, and $x \in \mathbb{R}^{(N+|V'|)}$. This is trivial even for $A$, because we can simply append $|V'|$ row or column pointers to the CSR/CSC representation.

\item Finally, we construct $A_u$ with a nonzero in every entry $(i,j)$ for each new edge between vertices $i$ and $j$. If we anticipate small, frequent updates, $A_u$ should be represented in unordered triplet format. Our new BFS step is thus $x_{i+1} = (A+A_u)x_i = Ax_i + A_ux_i = (A+A_u)^ix_0$. 
\end{enumerate}

\begin{align}\bordermatrix{ 
& & & j&  \cr
& & \infty & \vdots & \infty \cr
 i & &  \cdots & w & \cdots   \cr
  & & \infty & \vdots & \infty  \cr
   &  & & &  
} && \text{Update matrix $A_u$ contains nonzero at $A_u[i,j]$ for each new edge $(i,j)$. }\end{align}

Any shortest-path vectors already computed for $A$ must be recomputed from the beginning, because there is no trivial way to compute $(A+A_u)^i$ from $A^i$.

\subsection{Vertex, Edge Deletion}
Vertex deletion can be handled by deleting all edges in its row and column, and edge deletion by setting the corresponding nonzero in $A$ to 0. In the long-term $A$ should be consolidated, as this can be considered a memory leak. 

\subsection{Consolidation}
We consider the cost of combining $A+A_u$ into a single representation, in contrast to computing $Ax_i + A_ux_i$ over a particular number of iterations to give insight into when we should consolidate. 

A rule-of-thumb would be that we should consolidate when the following cost inequality is satisfied: \begin{align} cost(A + A_u) + \sum_{i=0}^N cost( A'x_i) \leq \sum_{i=0}^N [cost(Ax_i) + cost(A_ux_i)]\end{align}



\begin{align*}
\text{operation} && \text{flops} && \text{bytes} \\
A+A_u && 2(|E(A)|+|E(A_u)|) && (12�nnz +4�(n+1) +8�(n \cdots nnz) + 8�n) \\
A'x_i && 2(|E(A)|+|E(A_u)|) \\
Ax_ i && 2 |E(A)| \\
A_ux_i && 2|E(A_u)|
\end{align*}



%\begin{figure}
%\label{rpls}
%\center \includegraphics*[width=5in]{mat.png}
%\end{figure}

\bibliographystyle{plain}
\bibliography{bib}


\end{document}